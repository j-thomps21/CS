\documentclass{article}[9pt]

\usepackage{listings}
\usepackage{fullpage}
\usepackage{textcomp}
\usepackage{mdframed}

\lstset{ %
  basicstyle=\ttfamily\scriptsize,
  commentstyle=\ttfamily\scriptsize\emph,
  upquote=true,
  framerule=1.25pt,
  breaklines=true,
  showstringspaces=false,
  escapeinside={(*@}{@*)},
  belowskip=2em,
  aboveskip=1em,
}


\newenvironment{answerfont}{\fontfamily{qhv}\selectfont}{\par}
\newenvironment{myanswer}{\begin{mdframed}\begin{answerfont}}{\end{answerfont}\end{mdframed}}


\title{HW 13}

\author{Joshua Thompson}

\date{April 29, 2018}

\begin{document}

\maketitle
\section*{Questions}
\label{sec:org3ac93b9}

\begin{enumerate}

\item (15 points) For each of the statements, indicate if the statment is true or false. Add a brief statement to support the claim
\begin{enumerate}
  \item Threads are created just like processes by calling fork() except instead of checking the return value of fork() a specific funtion is used.
  \begin{myanswer}
  False, threads created using clone()
  \end{myanswer}

  \item Threads are scheduled just like other processes because POSIX threads are treated like individual processes by the OS.
  \begin{myanswer}
  False, they are treated as a single process
  \end{myanswer}

  \item Like multiple processes, threads provide resource isolation. Two threads from the same program do not share memory or other resources
  \begin{myanswer}
  False, threads share the same memory
  \end{myanswer}

  \item It's not possible for two threads of the same process to run simultaneously
  \begin{myanswer}
  False
  \end{myanswer}

  \item When any of the threads terminates, such as a call to exit(), all threads eliminate
  \begin{myanswer}
  True, threads all run on the same process
  \end{myanswer}
\end{enumerate}

\item (5 points) What are the equivalent thread commands for the following system calls:
\begin {enumerate}
  \item fork()
  \begin{myanswer}
  clone()
  \end{myanswer}

  \item wait()
  \begin{myanswer}
  pthread\_join()
  \end{myanswer}
\end{enumerate}

\item (15 points) Match the following terms, identifiers, funcitons to the descriptions below. (Online)
\begin{enumerate}
  \item Retrieve the POSIX thread identifier for the calling thread
  \begin{myanswer}
  pthread\_self()
  \end{myanswer}

  \item The process identifier, shared by all threads of a multi-threaded program
  \begin{myanswer}
  pid
  \end{myanswer}


  \item Retrieve the UNIX OS thread identifier of the calling thread
  \begin{myanswer}
  syscall(SYS\_gettid)
  \end{myanswer}


  \item Retrieve the UNIX OS process identifier of the calling process
  \begin{myanswer}
  getpid()
  \end{myanswer}

 \item the type of a POSIX thread identifier
  \begin{myanswer}
  pthread\_t
  \end{myanswer}

  \item The type of the UNIX OS thread identifier
  \begin{myanswer}
  pid\_t
  \end{myanswer}


  \item The thread identifier, unique to each thread and equal to the pid of the main thread
  \begin{myanswer}
  tid
  \end{myanswer}
\end{enumerate}

\item (5 points) Complete the program given online. The thread should print the command line argument passed to it.
\begin{myanswer}
\begin{verbatim}
void * startup(void * args){
  char * str = (char*)args;
  printf("%s", str);
  return NULL;
}

int main(int argc, char * argv[]){
  pthread_t thread;

  pthread_create(&thread, NULL, startup, argv[1]);

  pthread_join(thread, NULL);
  return 0;
}
\end{verbatim}
\end{myanswer}

\item Answer the following Questions about the given program online. You could assume this would be run on a lab machine; if you wanted to run it to answer the questions.
\begin{enumerate}
  \item (5 points) Based on the code, what are the two possible values for the argument to foo()?
  \begin{myanswer}
  1 or NULL
  \end{myanswer}

  \item (5 points) When you run this program, how many threads are running. You could use ps -l to count.
  \begin{myanswer}
  9
  \end{myanswer}

  \item (5 points) According to top what percent CPU does the program consume? Explain
  \begin{myanswer}
  900%
  \end{myanswer}

\end{enumerate}

\item (5 points) Explain why the following code snippet is not Atomic?
\begin{enumerate}
\begin{verbatim}
balance = balance+1;
\end{verbatim}
\end{enumerate}

\begin{myanswer}
During the addition the added part of the is stored in a temp variable before it is saved to the l value of balance. In between these actions the thread could try to access the variable and change it before the value can be saved to the l value.
The action can be interrupted.
\end{myanswer}

\item (5 points) For the code online, what is the expected output? Would you always get what you expect? Explain
\begin{myanswer}
"shared: 200" is expected but that will not always be the output because of the use of non atomic code. If scheduling changes, the output can become different.
\end{myanswer}

\item (5 points) For the code in the previous questions, identify the critical section. What makes this section critical?
\begin{myanswer}
shared++ because it is non atomic
\end{myanswer}

\item (5 points) Consider the naive locking solution shown online. Does this prove proper locking? Explain why or why not.
\begin{myanswer}
No because the the operations are not atomic. If the threads switch between the while command and the lock statement, there can be issues.
\end{myanswer}

\item (5 points) Explain why using mutex avoids issues of a lack of atomicity in lock acquisition?
\begin{myanswer}
Mutex avoids these issues becaue it is always guaranteed to be atomic during operation. The mutex itself is a lock and has functions for locking and unlocking.
\end{myanswer}

\item (10 points) The code shown online uses a coarse locking strategy, rewrite it to use fine locking.
\begin{myanswer}
\begin{verbatim}

pthread_mutext_t b_lock, c_lock;

int avail = MAX_FUNDS;
int local_1 = 0;
int local_2 = 0;

void * fun(void * args){
  int v,i;

  for(i=0; i < 100; i++){
    v = random() % 100;


    if(avail - v > 0){
      avail -= v;
    }

    if(random() % 2){
      pthread_mutext_lock(&b_lock);
      local_1 += v;
      pthread_mutext_unlock(&b_lock);
    }else{
      pthread_mutext_lock(&c_lock);
      local_2 += v;
      pthread_mutext_unlock(&c_lock);
    }

  }

  return NULL;
}
\end{verbatim}
\end{myanswer}

\item (5 points) What is deadlock? Provide a small code example of how deadlock can arise.
\begin{myanswer}
A deadlock is when two threads each hold a resource the other is waiting on. Example is the dining philosophers.
\end{myanswer}

\item (5 points) Explain a strategy to avoid deadlock.
\begin{myanswer}
A way to avoid deadlock is to order the way we give resources out to the threads so that they always have enough to continue.
\end{myanswer}


\end{enumerate}
\end{document}
