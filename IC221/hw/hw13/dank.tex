\documentclass{article}[9pt]

\usepackage{listings}
\usepackage{fullpage}
\usepackage{textcomp}
\usepackage{mdframed}

\lstset{ %
  basicstyle=\ttfamily\scriptsize,
  commentstyle=\ttfamily\scriptsize\emph,
  upquote=true,
  framerule=1.25pt,
  breaklines=true,
  showstringspaces=false,
  escapeinside={(*@}{@*)},
  belowskip=2em,
  aboveskip=1em,
}


\newenvironment{answerfont}{\fontfamily{qhv}\selectfont}{\par}
\newenvironment{myanswer}{\begin{mdframed}\begin{answerfont}}{\end{answerfont}\end{mdframed}}


\title{HW 11}

\author{Servando Varela}

\date{4/15/2018}

\begin{document}

\maketitle
\section*{Questions}
\label{sec:org3ac93b9}

\begin{enumerate}

\item 15 points) For each of the statements, indicate if the statement is true or false. You must add a brief statement to support your claim. 
  \begin{enumerate}
    \item Threads are created just like processes by calling fork() except instead of checking the return value of fork() a specific function is used.
      \begin{myanswer}
        False. Threads are created using clone().
      \end{myanswer}
    \item Threads are scheduled just like other processes because POSIX threads are treated like individual processes by the OS.
      \begin{myanswer}
        False. 
      \end{myanswer}
    \item Like multiple processes, threads provide resource isolation. Two threads from the same program do not share memory or other resources.
      \begin{myanswer}
        False. Threads share the same memory.
      \end{myanswer}
    \item It's not possible for two threads of the same process to run simultaneously.
      \begin{myanswer}
        True. Threads can run at the same time.
      \end{myanswer}
    \item When any of the threads terminates, such as a call to exit(), all threads terminate.
      \begin{myanswer}
        True. The threads are all running on the same process.
      \end{myanswer}
  \end{enumerate}

\item (5 points) What are the equivalent thread command for the following system calls:
  \begin{enumerate}
    \item fork()
      \begin{myanswer}
        clone()
      \end{myanswer}
    \item wait()
      \begin{myanswer}
        pthread\_join()
      \end{myanswer}
  \end{enumerate}


\item (15 points) Match the following terms, identifiers, functions to the descriptions below.
\begin{verbatim}
tid, pid, pid_t, pthread_t, syscall(SYS_getttid), getpid(), pthread_self()
\end{verbatim}
\begin{enumerate}

  \item Retrieve the POSIX thread identifier for the calling thread
    \begin{myanswer}
      pthread\_self()
    \end{myanswer}
  \item The process identifier, shared by all threads of a multi-threaded program.
    \begin{myanswer}
      pid
    \end{myanswer}
  \item Retrieve the UNIX OS thread identifier of the calling thread
    \begin{myanswer}
      syscall(SYS\_gettid)
    \end{myanswer}
  \item Retrieve the UNIX OS process identifier of the calling process
    \begin{myanswer}
      getpid()
    \end{myanswer}
  \item The type of a POSIX thread identifier
    \begin{myanswer}
      pthread\_t
    \end{myanswer}
  \item The type of the UNIX OS thread identfier
    \begin{myanswer}
      pid\_t
    \end{myanswer}
  \item The thread identifier, unique to each thread and equal to the pid of the main thread
    \begin{myanswer}
      tid
    \end{myanswer}
\end{enumerate}


\item (5 points) Complete the following program below. The thread should print the command line argument passed to it:
  \begin{verbatim}
void * startup( void * args){
  char * str = (char*) args; //varible to reference string to print
  printf("%s", str);
  return NULL;
}
int main(int argc, char * argv[]){
  pthread_t thread; //POSIX thread identifier

  //create a thread to run startup with argument argv[1]
  pthread_create(&thread, NULL, startup, argv[1]);

  pthread_join(thread, NULL);

  return 0;
}
  \end{verbatim}
  
      
\item Answer the following questions about the program below. You could assume this would be run on a lab machine, if you wanted to run it to answer the questions (hint!).
  \begin{verbatim}
#include <stdio.h>
#include <stdlib.h>
#include <pthread.h>

void * foo(void * args){
  pthread_t thread;

  if(args == NULL){
    pthread_create(&thread, NULL,foo, (void *) 1);
  }
  while(1);
}

int main(int argc, char * argv[]){
  pthread_t threads[4];
  int i;

  for(i=0;i<4;i++){
    pthread_create(&threads[i], NULL,foo, NULL);
  }
  while(1);
}
  \end{verbatim}
  \begin{enumerate}
    \item (5 points) Based on the code, what are the two possible values for the argument to foo()?
      \begin{myanswer}
      \end{myanswer}
    \item (5 points) When you run this program, how many threads are running. You could use ps -L to count.
      \begin{myanswer}
      \end{myanswer}
    \item (5 points) According to top what percent CPU does the program consume? Explain.
      \begin{myanswer}
      \end{myanswer}
    \end{enumerate}

\item (5 points) Explain why the following code snippet is not atomic?
\begin{verbatim}
balance = balance+1
\end{verbatim}
\begin{myanswer}
\end{myanswer}

\item (5 points) For the code below, what is the expected output? Would you always get what you expect? Explain.
\begin{verbatim}
int shared = 0;

void * fun(void * args){
  int i;
  for(i=0;i<100;i++){
    shared++;
  }
  return NULL;
}
int main(){
  pthread_t t1,t2;
  pthread_create(&t1, NULL, fun, NULL);
  pthread_create(&t2, NULL, fun, NULL);

  pthread_join(t1, NULL);
  pthread_join(t2, NULL);

  printf("shared: %d\n", shared);
}
\end{verbatim}  
\begin{myanswer}
  \end{myanswer}

\item (5 points) For the code in the previous question, identify the critical section. What makes this section critical?
\begin{myanswer}
  \end{myanswer}

\item (5 points) Consider the naive locking solution below. Does this prove proper locking? Explain why or why not.
\begin{verbatim}
int shared;
int lock;

void * fun(void * args){
  int i;

  for(i=0;i<100;i++){
    while(lock > 0);//spin
    lock = 1; //set lock
    shared++; //increment
    lock = 0; //unlock
  }
  return NULL;
}
\end{verbatim}
\begin{myanswer}
  \end{myanswer}

\item (5 points) Explain why using a mutex avoids issues of a lack of atomicity in lock acquisition?
\begin{myanswer}
  \end{myanswer}

\item (10 points) The cod below uses a course locking strategy, rewrite it to use fine locking.
	\begin{verbatim}
pthread_mutext_t lock;
int avail = MAX_FUNDS;
int local_1 = 0;
int local_2 = 0;

void * fun(void * args){
  int v,i;

  for(i=0; i < 100; i++){
    v = random() % 100;

    pthread_mutext_lock(&lock);

    if(avail - v > 0){
      avail -= v;
    }
    if(random() % 2){
      local_1 += v;
    }else{
      local_2 += v;
    }
    pthread_mutext_unlock(&lock);
  }
  return NULL;

}
	\end{verbatim}
\begin{myanswer}
\end{myanswer}

\item (5 points) What is deadlock? Provide a small, (pseudo-) code example of how deadlock can arise.
\begin{myanswer}
\end{myanswer}

\item (5 points) Explain a strategy for avoiding deadlock.
\begin{myanswer}
\end{myanswer}


\end{enumerate}
\end{document}
