
\documentclass{article}[9pt]

\usepackage{listings}
\usepackage{fullpage}
\usepackage{textcomp}

\lstset{ %
  basicstyle=\ttfamily\small,
  commentstyle=\ttfamily\small\emph,
  upquote=true,
  framerule=1.25pt,
  breaklines=true,
  showstringspaces=false,
  escapeinside={(*@}{@*)},
  belowskip=2em,
  aboveskip=1em,
}


\newcommand{\myanswer}[1]{\framebox[.9\linewidth]{\parbox{.85\linewidth}{\fontfamily{qhv}\selectfont{#1}}}}

\title{HW 2}

\author{Joshua Thompson}

\date{22 JAN 2018}

\begin{document}

\maketitle

\begin{enumerate}

\item (10 points) Consider the \texttt{ls -l} output below, label the output
appropriately:

\begin{verbatim}
drwxr-xr-x 2 aviv scs 4096 2013-12-22 10:57 demo/
-rw-r--r-- 1 aviv scs 13454 2013-12-22 10:56 text.dat
\end{verbatim}

\item (10 points) Describe the permission strings below in plain English? What
permission does the owner have, group have, and other have?

\begin{enumerate}

\item \texttt{rwxrw-rw-}

      \myanswer{Owner: read, write execute. Group: read, write. Global: read, write.}

\item \texttt{r-x-{}-x-{}-x}

      \myanswer{Owner: read, execute. Group: execute. Global: execute.}

\item \texttt{rw-rw-r-{}}

      \myanswer{Owner: read, write. Group: read, write. Global: read.}

\end{enumerate}

\item (20 points) Using the man page (or \textbf{trying it yourself in the terminal})
describe the resulting action of the following \texttt{chmod} commands.

\begin{enumerate}

\item \texttt{chmod a+x file}

      \myanswer{Gives all users (User, group, and global) execute privileges.}

\item \texttt{chmod u+x file}

      \myanswer{Gives the user execute privileges.}

\item \texttt{chmod a-w file}

      \myanswer{Takes away all write privileges.}

\item  \texttt{chmod g+rw file}

      \myanswer{Gives group read and write privileges}

\end{enumerate}

\item (10 points) On a lab computer, type \texttt{groups} into the shell. What groups are
you in?

      \myanswer{ I am in the "cfy gp mids" group }

\item (10 points) In what configuration is the following information found?

  \begin{enumerate}

\item The list of all users on the systems?

      \myanswer{Stated in the /etc/passwd file. Last name, First name, Midn, USN, etc}

\item The default group of a user?

      \myanswer{Question not worded well. Unsure of what it is asking}

\item The list of all groups and members (perhaps not the default group?)

      \myanswer{Question not worded well. Unsure of what it is asking}

\end{enumerate}

\item (20 points) On a lab computer, run \texttt{\textasciitilde{}aviv/bin/ic221-up} and change into the
\texttt{hw/02} directory. There is a program called \texttt{runme}. Change the
permission of this program such that is executable and run it.

\begin{enumerate}

\item What command did you use to run it?

      \myanswer{I used the ./ command to run the program}
\item What is the output of running the command?

      \myanswer{The output is 4 cat looking things. With the middle two looking at each other.}

\end{enumerate}
\item (20 points) Use \texttt{ls} to list the directory \texttt{\textasciitilde{}aviv/ic221-hw/hw02/} where
you'll find a secret file called \texttt{secret} that you \textbf{do not} have
permission to execute.

\begin{enumerate}

\item Explain how you could still execute this program despite not having group
  or global execute permissions?

      \myanswer{I copied the secret program into my own directory which in turn made me the owner of the file}

\item Do so, and what is the output?

      \myanswer{The output is a submarine}

\end{enumerate}

\end{enumerate}

\end{document}


%%% Local Variables:
%%% mode: latex
%%% TeX-master: t
%%% End:
