\documentclass{article}[9pt]

\usepackage{listings}
\usepackage{fullpage}
\usepackage{textcomp}
\usepackage{mdframed}

\lstset{ %
  basicstyle=\ttfamily\scriptsize,
  commentstyle=\ttfamily\scriptsize\emph,
  upquote=true,
  framerule=1.25pt,
  breaklines=true,
  showstringspaces=false,
  escapeinside={(*@}{@*)},
  belowskip=2em,
  aboveskip=1em,
}


\newenvironment{answerfont}{\fontfamily{qhv}\selectfont}{\par}
\newenvironment{myanswer}{\begin{mdframed}\begin{answerfont}}{\end{answerfont}\end{mdframed}}


\title{HW 11}

\author{Joshua Thompson}

\date{April 14, 2018}

\begin{document}

\maketitle
\section*{Questions}
\label{sec:org3ac93b9}

\begin{enumerate}
\item (5 points) What are the two parts of a packet? Which stores the address and what stores the data?

\begin{myanswer}
The header which stores the address and the payload which stores the data
\end{myanswer}


\item (5 points) What does it mean for the Internet to be "packet switched"?

  \begin{myanswer}
  It means that when the packet arrives at routers or switches, based solely on the header, the device knows where to send the packet.
  \end{myanswer}

\item (5 points) What are the different layers of the protocol stack and what purpose do they serve in delivering a packet?


  \begin{myanswer}
  The layers are the Physical, link, Internet, Transport, and Application layers. The physical layer's purpose is to send signals accross a wire. The Link adds protocols for how the medium (a wire) shares signals across many devices. The Internet and
  transport layers interact with the protocols from the Link layer. Internet purpose is to inter-connect networks via the internet protocol address aka ip address. The transport layer allows two computers across the internet to act as if they were talking
  directly to each other. The application layer can add further protocols based on the functionality of the application itself.
  \end{myanswer}

\item (5 points) An Internet address (version 4) is stored in how many bytes?

  \begin{myanswer}
  4 bytes.
  \end{myanswer}


\item (5 points) A domain name, unlike an IP address, is more human usable, what protocol enables domain names to be resolved into IP addresses?

    \begin{myanswer}
    DNS
    \end{myanswer}


\item (15 points) Using the host command line tool, resolve the following domain names to an IPv4 address. Indicate those that also have IPv6 addresses, and also, rerun host a few times,
 indicate which hosts IP addresses changed on subsequent runs.

  \begin{enumerate}
  \item \texttt{www.cis.upenn.edu}
    \begin{myanswer}
    \begin{verbatim}
    IPv4: 158.130.69.163
    IPv6: 2607:f470:8:64:5ea5::d
    \end{verbatim}
    \end{myanswer}

  \item \texttt{www.cs.swarthmore.edu}
    \begin{myanswer}
    \begin{verbatim}
    IPv4: 130.58.68.137
    \end{verbatim}
    \end{myanswer}

  \item \texttt{www.usna.edu}
    \begin{myanswer}
    \begin{verbatim}
    IPv4: 10.4.36.20
    \end{verbatim}
    \end{myanswer}

  \item \texttt{facebook.com}
    \begin{myanswer}
    \begin{verbatim}
    IPv4: 31.13.69.228
    IPv6: 2a03:2880:f103:83:face:b00c:0:25de
    \end{verbatim}
    \end{myanswer}

  \item \texttt{microsoft.com}
    \begin{myanswer}
    \begin{verbatim}
    IPv4: 104.43.195.251
    IPv4: 23.100.122.175
    IPv4: 23.96.52.53
    IPv4: 191.239.213.197
    IPv4: 104.40.211.35
    Microsoft changes IP addresses when running host multiple times.
    \end{verbatim}
    \end{myanswer}

  \end{enumerate}

\item (10 points) What is the purpose of the port address? How many bytes and what C type would naturally store a port address?

  \begin{myanswer}
  The port address tells the packet which process to go to, within the computer. It is a two-byte unsigned short
  \end{myanswer}

\item (10 points) TCP provides reliable data transmission, but at what cost? Why might you want to use UDP over TCP?

  \begin{myanswer}
  The cost is that in order for the protocol to be reliable, there must be confirmation of recieving packets, and retransmission of lost packets. Therefore, there is a lot
  of overhead while using this protocol. One might want to use UDP instead of TCP if they don't mind losing some packets here and there. An example is audio streaming.
  \end{myanswer}

\item (20 points) For each of the descriptions below of a network type, indicate the type that best matches that description.

  \item \texttt{Specifies the address type, eg AF_INET, for the addrinfo structure}
    \begin{myanswer}\begin{verbatim}
    ai_family
    \end{verbatim}\end{myanswer}

  \item \texttt{Specifies the address type for sockaddr structure}
    \begin{myanswer}\begin{verbatim}
    sin_family
    \end{verbatim}\end{myanswer}

  \item \texttt{A type defined as a uint32}
    \begin{myanswer}\begin{verbatim}
    s_addr
    \end{verbatim}\end{myanswer}

  \item \texttt{A generic address structure for sockets}
    \begin{myanswer}\begin{verbatim}
    struct saddr
    \end{verbatim}\end{myanswer}

  \item \texttt{A structure to store an IPv4 internet address}
    \begin{myanswer}\begin{verbatim}
    in_addr
    \end{verbatim}\end{myanswer}

  \item \texttt{An unsigned short storing the port for a sockaddr_in}
    \begin{myanswer}\begin{verbatim}
    sin_port
    \end{verbatim}\end{myanswer}

  \item \texttt{Structure used to hint at IP addresses for resolving as well as storing results}
    \begin{myanswer}\begin{verbatim}
    addrinfo
    \end{verbatim}\end{myanswer}

  \item \texttt{Member of the sockaddr_in that stores the address}
    \begin{myanswer}\begin{verbatim}
    struct in_addr
    \end{verbatim}\end{myanswer}

  \item \texttt{The sole member of the in_addr structure}
    \begin{myanswer}\begin{verbatim}
    s_addr
    \end{verbatim}\end{myanswer}

  \item \texttt{A generic socket address returned in addrinfo}
    \begin{myanswer}\begin{verbatim}
    struct sockaddr
    \end{verbatim}\end{myanswer}

  \item \texttt{A specific address structure for sockets to store IP, port pairs}
    \begin{myanswer}\begin{verbatim}
    sockaddr
    \end{verbatim}\end{myanswer}

\item (5 points) The following functions are opposites, inet_ntoa() and inet_aton(), what are their purposes? Give example.

  \begin{myanswer}
  \begin{verbatim}
  inet_ntoa() is used to convert a struct into an IP address. inet_aton() does the opposite.
  \end{verbatim}
  \end{myanswer}

\item (5 points) When assign a port to a socket address, which of these two conversion should you use and why? htnos() vs ntohs().

  \begin{myanswer}
  We use htons() because of network byte order.
  \end{myanswer}

\item (10 points) Consider setting the address 10.4.32.41 on port 22. Complete the code below to do that.

\begin{enumerate}
\begin{verbatim}
struct sockaddr_in saddr;
//TODO: what assignments come next
\end{verbatim}
\end{enumerate}

\begin{myanswer}
\begin{verbatim}
struct: sockaddr_in saddr;

saddr = (struct sockaddr_in*) result->ai_addr;
saddr->sin_port = htnos(80);
\end{verbatim}
\end{myanswer}


\end{enumerate}
\end{document}
