\documentclass{article}[9pt]

\usepackage{listings}
\usepackage{fullpage}
\usepackage{textcomp}
\usepackage{mdframed}

\lstset{ %
  basicstyle=\ttfamily\scriptsize,
  commentstyle=\ttfamily\scriptsize\emph,
  upquote=true,
  framerule=1.25pt,
  breaklines=true,
  showstringspaces=false,
  escapeinside={(*@}{@*)},
  belowskip=2em,
  aboveskip=1em,
}


\newenvironment{answerfont}{\fontfamily{qhv}\selectfont}{\par}
\newenvironment{myanswer}{\begin{mdframed}\begin{answerfont}}{\end{answerfont}\end{mdframed}}


\title{HW 6}

\author{Joshua Thompson}

\date{27 FEB 2017}

\begin{document}

\maketitle

\section*{Questions}
\label{sec:orgd8f1586}

\begin{enumerate}
\item (5 points) Why does the operating system perform system calls as
oppose to having each user perform the same operations
themselves?

\begin{myanswer}
Because the user could do damage to the operating system if they write code at such a low level.
\end{myanswer}


\item (10 points) Look up the following C functions in the man page, label them as
either a system call or not a system call.

\begin{myanswer}
write your answer here
\end{myanswer}

\begin{enumerate}
\item fread()

\begin{myanswer}
not a system call
\end{myanswer}

\item write()

\begin{myanswer}
system call
\end{myanswer}

\item stat()

\begin{myanswer}
system call
\end{myanswer}


\item mmap()

\begin{myanswer}
system call
\end{myanswer}


\item execv()

\begin{myanswer}
not a system call
\end{myanswer}

\end{enumerate}

\item (10 points) Run \texttt{ic221-up}. In the \texttt{hw/06/prob3} directory you'll find a compiled
program. Use \texttt{ltrace} to enumerate the library function calls
occurring under \texttt{main()}.

\begin{myanswer}
The program uses the strlen, puts, fflush and exit library calls.
\end{myanswer}

\item (10 points) Run \texttt{ic221-up}. In the \texttt{hw/06/prob4} directory you'll find a compiled
program. Use \texttt{strace} to enumerate the system function calls
occurring under \texttt{main()}.

\begin{myanswer}
There are several system calls being used. brk, openat, fstat, read, fstat, mmap, mprotect, close, stat, write.
\end{myanswer}

\item (20 points) Consider a file, \texttt{accts.dat}, which stores 1000 accounts
formatted based on the defined structure. Using \texttt{open()} and
\texttt{read()}, complete the program below to print them out:


\begin{lstlisting}[language=c]
#include <stdio.h>
#include <stdlib.h>
#include <unistd.h>
#include <fcntl.h>

typedef struct{
  long acctnum;
  double bal;
  char acctname[1024];
} acct_t;

void read_accts(accts){
  //COMPLETE ME
}

int main(int argc, char *argv[]){
  acct_t accts[1000];

  read_accts(accts);

  int i;
  for(i=0;i<1000;i++){
    printf("%ld (%f) -- %s\n",
           accts[i].acctnum,
           accts[i].bal,
           accts[i].acctname);
  }

  close(fd);

}
\end{lstlisting}

\begin{myanswer}
\begin{lstlisting}[language=c]
#include <stdio.h>
#include <stdlib.h>
#include <unistd.h>
#include <fcntl.h>

typedef struct
{
  long acctnum;
  double bal;
  char acctname[1024];
} acct_t;

void read_accts(acct_t** accts, char* fname)
{
  int fd;
  fd = open(fname, O_RDONLY);
  read(fd, &accts, 1000*sizeof(acct_t));
  close(fd);
}

int main(int argc, char *argv[]) {
  acct_t* accts;
  read_accts(&accts, argv[1]);

  int i;
  for(i = 0; i < 1000; i++)
  {
    printf("%ld (%f) -- %s\n", accts[i].acctnum, accts[i].bal, accts[i].acctname);
  }
  return 0;
}
\end{lstlisting}
\end{myanswer}

\item (10 points) Complete the following ORing options that matching the \texttt{fopen()} permissions:
\begin{enumerate}
\item \texttt{r}

\begin{myanswer}
O_RONLY
\end{myanswer}

\item \texttt{w}

\begin{myanswer}
O_WRONLY
\end{myanswer}

\item \texttt{a}

\begin{myanswer}
O_APPEND
\end{myanswer}

\item \texttt{w+}

\begin{myanswer}
O_CREATE
\end{myanswer}

\item \texttt{r+}

\begin{myanswer}
O_APPEND
\end{myanswer}

\end{enumerate}

\item (10 points) Consider a \texttt{umask} of 0750 (the leading 0 is to indicate a number
written in octal). For the following \texttt{open()} permissions, what
actual permission will the file get? You can write your answers in
octal.

\begin{myanswer}
write your answer here
\end{myanswer}


\begin{enumerate}
\item 0777

\begin{myanswer}
020
\end{myanswer}

\item 0640

\begin{myanswer}
000
\end{myanswer}

\item 0740

\begin{myanswer}
000
\end{myanswer}

\item 0501

\begin{myanswer}
001
\end{myanswer}

\item 0651
\end{enumerate}

\item (5 points) Explain why the umask is considered a security feature.

\begin{myanswer}
umask ensures that we do not create a file with more permissions than what we want. 
\end{myanswer}

\end{enumerate}
\end{document}
